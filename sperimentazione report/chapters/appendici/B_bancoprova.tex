\section{Presentazione banco prova}\label{app:banco}
Si riporta lo schema del banco prova utilizzato per le seguenti misure:
\begin{itemize}
	\item misure di temperatura mediante termocoppia, inserita nella fiamma del
	bruciatore a gas naturale (Sezione \ref{sez:misureT});
	\item misure di \gls{symb:cd} per caratterizzare l'ugello del miscelatore (Sezione \ref{sez:cd}).
\end{itemize}

Il banco prova ha come obiettivo la realizzazione di misure di temperatura in una fiamma generata dalla combustione di una miscela di aria e gas naturale, in condizioni prossime a quelle stechiometriche. 
Il gas passa per un misuratore di portata a galleggiante, poi per una valvola utilizzata per regolarne la portata. Successivamente è presente una camera di stanca, utilizzata per rallentare il flusso e aumentarne la pressione. Il gas poi passa per un eiettore, che permette di aspirare la portata d'aria necessaria per la combustione stechiometrica. Il flusso viene poi inviato a un bruciatore, dove avviene la combustione. Infine i gas vengono scaricati in atmosfera. 

Gli strumenti di misura presenti sono i seguenti:
\begin{itemize}
	\item termocoppia di tipo B (per la misura di temperatura), connessa a una scheda di acquisizione dati;
	\item termocoppia in camera di stanca;
	\item misuratore di pressione differenziale tra camera di stanca e ambiente esterno;
	\item misuratore di portata a galleggiante.
\end{itemize}
\begin{figure}[H]
	\centering
	\includegraphics[width=\linewidth]{"D:/repositoriesgithub/sperimentazione-propulsori/sperimentazione report/chapters/appendici/banco"}
	\caption{Schema del banco prova}
	\label{fig:banco}
\end{figure}

