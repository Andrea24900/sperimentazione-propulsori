\section{Misure di temperatura mediante termocoppia}
\paragraph{Dati e richieste}
Vengono fornite due serie di misure di temperatura allo scarico di una camera di combustione, eseguite mediante termocoppia di tipo B. La prima è costituita da 1599 valori ("Serie corta"), la seconda da 9999 ("Serie lunga"). Entrambe le serie sono campionate con una frequenza di campionamento di 100 Hz e vengono fornite mediante file testuale (.txt).\\
Si chiede di svolgere l'analisi statistica dei dati.
\subsection{Risoluzione}
Si riportano i risultati emersi dall'elaborazione dei dati sperimentali. I calcoli sono stati svolti mediante il software \textit{Matlab} e le funzioni built-in.
\paragraph{Suddivisione in classi e istogramma}
Entrambe le serie sono divise in 10 classi, di uguale ampiezza, costruite affinché non ci possa essere ambiguità nell'attribuzione dei valori: poiché le misure hanno 6 cifre decimali, gli estremi di classe sono definiti con 7 cifre decimali. L'estremo della prima classe viene scelto come il minimo valore di \glsxtrshort{symb:T} a cui viene sottratto 0.5e-7 \textsuperscript{o}C. Analogamente, l'estremo superiore dell'ultima classe viene calcolato sommando la stessa quantità al massimo valore di \glsxtrshort{symb:T} nella serie. I valori estremi delle due serie sono mostrati in Tab.\ref{tab:estremitemp}, riportati integralmente per mettere in evidenza il numero di cifre decimali.
\begin{table} [H]
	\centering
	\begin{tabular}{c|c|c}
		\toprule
		\toprule
		\textbf{Serie} & \textbf{T\textsubscript{min} [\textsuperscript{o}C]} &\textbf{T\textsubscript{max} [\textsuperscript{o}C]} \\
		\midrule
		\midrule
		Corta & 953.745910 & 1193.110960\\
		\midrule
		Lunga & 931.352290 & 1449.917970 \\
		\bottomrule
		\bottomrule
	\end{tabular}
\caption{Valori estremi di temperatura}
\label{tab:estremitemp}
\end{table}
Gli istogrammi relativi alle due serie sono mostrati in Fig.\ref{fig:istboth}.
\begin{figure} [H]
	
	\begin{subfigure}{0.49\textwidth}
		\centering
		\includegraphics[width=0.99\linewidth]{chapters/1-misureT/istogrammashort}
		\caption{Serie corta}
		\label{fig:istogrammashort}
	\end{subfigure}%
	\begin{subfigure}{0.49\textwidth}
		\centering
		\includegraphics[width=0.99\linewidth]{chapters/1-misureT/istogrammalong}
		\caption{Serie lunga}
		\label{fig:istogrammalong}
	\end{subfigure}
\caption{Istogrammi delle due serie}	
\label{fig:istboth}
\end{figure}
Si osserva come entrambe le distribuzioni di dati siano simili alla distribuzione gaussiana, mostrando tuttavia una evidente asimmetria. Quest'ultima è quantificabile dal coefficiente di skewness, riportato in Tab.%\ref{}.
\paragraph{Calcolo delle frequenze relative e cumulate}
Successivamente vengono riportate in Tab.\ref{tab:seriecorta} e Tab.\ref{tab:serielunga}
le classi, il numero di valori in ciascuna di esse, le frequenze relative (\glsxtrshort{symb:f}) e frequenze cumulate normalizzate(\glsxtrshort{symb:F}).

\begin{table}[H]
	\centering
	\begin{tabular}{c|c|c|c|c}
		\toprule
		\toprule
		\textbf{Classe}&\textbf{Estremi} &\textbf{Occorrenze}&\textbf{\glsxtrshort{symb:f}}&\textbf{\glsxtrshort{symb:F}}\\
		\midrule
		\midrule
		\multirow{2}{*}{1}&43535&\multirow{2}{*}{}&\multirow{2}{*}{}&\multirow{2}{*}{} \\
		&s74747&&&\\
		\midrule
		\multirow{2}{*}{2}&43535&\multirow{2}{*}{}&\multirow{2}{*}{}&\multirow{2}{*}{} \\
		&s74747&&&\\
		\midrule
		\multirow{2}{*}{3}&43535&\multirow{2}{*}{}&\multirow{2}{*}{}&\multirow{2}{*}{} \\
		&s74747&&&\\
		\midrule
		\multirow{2}{*}{4}&43535&\multirow{2}{*}{}&\multirow{2}{*}{}&\multirow{2}{*}{} \\
		&s74747&&&\\
		\midrule
		\multirow{2}{*}{5}&43535&\multirow{2}{*}{}&\multirow{2}{*}{}&\multirow{2}{*}{} \\
		&s74747&&&\\
		\midrule
		\multirow{2}{*}{6}&43535&\multirow{2}{*}{}&\multirow{2}{*}{}&\multirow{2}{*}{} \\
		&s74747&&&\\
		\midrule
		\multirow{2}{*}{7}&43535&\multirow{2}{*}{}&\multirow{2}{*}{}&\multirow{2}{*}{} \\
		&s74747&&&\\
		\midrule
		\multirow{2}{*}{8}&43535&\multirow{2}{*}{}&\multirow{2}{*}{}&\multirow{2}{*}{} \\
		&s74747&&&\\
		\midrule
		\multirow{2}{*}{9}&43535&\multirow{2}{*}{}&\multirow{2}{*}{}&\multirow{2}{*}{} \\
		&s74747&&&\\
		\midrule
		\multirow{2}{*}{10}&43535&\multirow{2}{*}{}&\multirow{2}{*}{}&\multirow{2}{*}{} \\
		&s74747&&&\\
		\midrule
		\bottomrule
		\bottomrule
	\end{tabular}
\caption{Risultati relativi alla serie corta}
\label{tab:seriecorta}
\end{table} 
\begin{table}[H]
	\centering
	\begin{tabular}{c|c|c|c|c}
		\toprule
		\toprule
		\textbf{Classe}&\textbf{Estremi} &\textbf{Occorrenze}&\textbf{\glsxtrshort{symb:f}}&\textbf{\glsxtrshort{symb:F}}\\
		\midrule
		\midrule
		\multirow{2}{*}{1}&43535&\multirow{2}{*}{}&\multirow{2}{*}{}&\multirow{2}{*}{} \\
		&s74747&&&\\
		\midrule
		\multirow{2}{*}{2}&43535&\multirow{2}{*}{}&\multirow{2}{*}{}&\multirow{2}{*}{} \\
		&s74747&&&\\
		\midrule
		\multirow{2}{*}{3}&43535&\multirow{2}{*}{}&\multirow{2}{*}{}&\multirow{2}{*}{} \\
		&s74747&&&\\
		\midrule
		\multirow{2}{*}{4}&43535&\multirow{2}{*}{}&\multirow{2}{*}{}&\multirow{2}{*}{} \\
		&s74747&&&\\
		\midrule
		\multirow{2}{*}{5}&43535&\multirow{2}{*}{}&\multirow{2}{*}{}&\multirow{2}{*}{} \\
		&s74747&&&\\
		\midrule
		\multirow{2}{*}{6}&43535&\multirow{2}{*}{}&\multirow{2}{*}{}&\multirow{2}{*}{} \\
		&s74747&&&\\
		\midrule
		\multirow{2}{*}{7}&43535&\multirow{2}{*}{}&\multirow{2}{*}{}&\multirow{2}{*}{} \\
		&s74747&&&\\
		\midrule
		\multirow{2}{*}{8}&43535&\multirow{2}{*}{}&\multirow{2}{*}{}&\multirow{2}{*}{} \\
		&s74747&&&\\
		\midrule
		\multirow{2}{*}{9}&43535&\multirow{2}{*}{}&\multirow{2}{*}{}&\multirow{2}{*}{} \\
		&s74747&&&\\
		\midrule
		\multirow{2}{*}{10}&43535&\multirow{2}{*}{}&\multirow{2}{*}{}&\multirow{2}{*}{} \\
		&s74747&&&\\
		\midrule
		\bottomrule
		\bottomrule
	\end{tabular}
	\caption{Risultati relativi alla serie lunga}
	\label{tab:serielunga}
\end{table} 


\begin{figure}
	
	\begin{subfigure}{0.5\textwidth}
		\centering
		\includegraphics[width=\linewidth]{chapters/1-misureT/relshort}
		\caption{Serie corta}
		\label{fig:relshort}
	\end{subfigure}%
	\begin{subfigure}{0.5\textwidth}
		\centering
		\includegraphics[width=\linewidth]{chapters/1-misureT/rellong}
		\caption{Serie lunga}
		\label{fig:rellong}
	\end{subfigure}
	\caption{Frequenze relative delle diverse classi}
	\label{fig:relboth}
\end{figure}

\begin{figure}
	\centering
	\includegraphics[width=0.7\linewidth]{chapters/1-misureT/cumboth}
	\caption{Frequenze cumulate normalizzate per entrambe le serie}
	\label{fig:cumboth}
\end{figure}