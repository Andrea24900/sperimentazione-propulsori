\section{Misure di temperatura mediante termocoppia} \label{sez:misureT}
\paragraph{Dati e richieste}
Vengono fornite due serie di misure di temperatura allo scarico di una camera di combustione, eseguite mediante termocoppia di tipo B. La prima è costituita da 1599 valori ("Serie corta"), la seconda da 9999 ("Serie lunga"). Entrambe le serie sono campionate con una frequenza di campionamento di 100 Hz e vengono fornite mediante file testuale (.txt).\\
Si chiede di svolgere l'analisi statistica dei dati.
\subsection{Risoluzione e analisi statistica}
Si riportano i risultati emersi dall'elaborazione dei dati sperimentali. I calcoli sono stati svolti mediante il software \textit{Matlab} e le funzioni built-in.
\paragraph{Suddivisione in classi e istogramma}
Entrambe le serie sono divise in 10 classi, di uguale ampiezza, costruite affinché non ci possa essere ambiguità nell'attribuzione dei valori: poiché le misure hanno 6 cifre decimali, gli estremi di classe sono definiti con 7 cifre decimali. L'estremo della prima classe viene scelto come il minimo valore di \gls{symb:T} a cui viene sottratto 0.5e-7 \textsuperscript{o}C. Analogamente, l'estremo superiore dell'ultima classe viene calcolato sommando la stessa quantità al massimo valore di \gls{symb:T} nella serie. I valori estremi delle due serie sono mostrati in Tab.\ref{tab:estremitemp}, riportati integralmente per mettere in evidenza il numero di cifre decimali.
\begin{table} [H]
	\centering
	\begin{tabular}{c|c|c}
		\toprule
		\toprule
		\textbf{Serie} & \textbf{\textit{T\textsubscript{MIN}} [\textsuperscript{o}C]} &\textbf{\textit{T\textsubscript{MAX}} [\textsuperscript{o}C]} \\
		\midrule
		\midrule
		Corta & 953.745910 & 1193.110960\\
		\midrule
		Lunga & 931.352290 & 1449.917970 \\
		\bottomrule
		\bottomrule
	\end{tabular}
\caption{Valori estremi di temperatura}
\label{tab:estremitemp}
\end{table}
Gli istogrammi relativi alle due serie sono mostrati in Fig.\ref{fig:istboth}.


\begin{figure} [H]
	\centering
	\begin{subfigure}{0.49\textwidth}
		\centering
		\includegraphics[width=0.99\linewidth]{"../sperimentazione nei propulsori/istogrammashort"}
		\caption{Serie corta}
		\label{fig:istogrammashort}
	\end{subfigure}%
	\begin{subfigure}{0.49\textwidth}
		\centering
		\includegraphics[width=0.99\linewidth]{"../sperimentazione nei propulsori/istogrammalong"}
		\caption{Serie lunga}
		\label{fig:istogrammalong}
	\end{subfigure}
\caption{Istogrammi delle due serie}	
\label{fig:istboth}
\end{figure}
Si osserva come entrambe le distribuzioni di dati siano simili alla distribuzione gaussiana, mostrando tuttavia una evidente asimmetria. Quest'ultima è quantificabile dal coefficiente di skewness, riportato in Tab.\ref{tab:indicistat}.

\paragraph{Calcolo delle frequenze relative e cumulate}
Successivamente vengono riportate delle rappresentazioni grafiche 
delle frequenze relative (\gls{symb:f}), in Fig.\ref{fig:relboth}, e delle frequenze cumulate normalizzate(\gls{symb:F}) delle varie classi, in Fig. \ref{fig:cumboth}. I risultati numerici sono riportati in Appendice \ref{app:statistica}.



\begin{figure}
	\centering
	\includegraphics[width=0.7\linewidth]{"../sperimentazione nei propulsori/relboth"}
	\caption{Frequenze relative per entrambe le serie}
	\label{fig:relboth}
\end{figure}


\begin{figure}
	\centering
	\includegraphics[width=0.7\linewidth]{"../sperimentazione nei propulsori/cumboth"}
	\caption{Frequenze cumulate normalizzate per entrambe le serie}
	\label{fig:cumboth}
\end{figure}



\paragraph{Calcolo degli indici statistici}
L'analisi statistica dei dati viene svolta mediante il calcolo degli indici statistici relativi alle due serie di dati. In particolare, si riportano media (\gls{symb:Tmean}) e mediana (\gls{symb:Tmed}) delle due serie, nonché deviazione standard (\gls{symb:sigmaT}) e skewness (\gls{symb:skew}) delle distribuzioni. I risultati sono presentati in Tab.\ref{tab:indicistat}.

\begin{table} [H]
	\centering
	\begin{tabular}{c|c|c}
		\toprule
		\toprule
		\textbf{Indice} & \textbf{Serie corta}&\textbf{Serie lunga}\\
		\midrule
		\midrule
		$\overline{T}$ [\textsuperscript{o}C]& 1082.8 & 1159.2\\
		\midrule
		$T_\textit{MEDIANA}$ [\textsuperscript{o}C]&1085.7&1152.4\\
		\midrule
		$\sigma_T$ [\textsuperscript{o}C]&39.147 & 77.841\\
		\midrule
		$SK$ & -0.27 & 0.40 \\
		\bottomrule
		\bottomrule
	\end{tabular}
\caption{Indici statistici delle due distribuzioni}
\label{tab:indicistat}
\end{table}

\paragraph{Stima dell'errore statistico}
Lo studio statistico delle due serie di dati si conclude con la stima dell'errore statistico (\gls{symb:Estat}).  Risulta necessario calcolare la deviazione standard del valore medio di temperatura (\gls{symb:sigmaTmean}) secondo:
\begin{equation}
	\sigma_{\overline{T}} = \frac{\sigma_{T}}{\sqrt{N}} \label{eq:sigmaTmean}
\end{equation}
dove \gls{symb:N} è il numero di campioni di ciascuna serie.
Infine, il valore di \gls{symb:Estat} si ottiene con:
\begin{equation}
	\epsilon_{\textit{STAT}}=\sigma_{\overline{T}}\,t_{\textit{95\%}}
\end{equation}
dove \gls{symb:t95} è ricavato dalla distribuzione t per un intervallo di confidenza al 95\% (dove \gls{symb:nu} = \gls{symb:N} - 1, dove 1 rappresenta il numero di gradi di libertà persi a seguito dell'introduzione di \gls{symb:Tmean}).
\begin{table} [H]
	\centering
	\begin{tabular}{c|c|c}
		\toprule
		\toprule
		\textbf{Indice} & \textbf{Serie corta}&\textbf{Serie lunga}\\
		\midrule
		\midrule
		$t_{\textit{95\%}}$ &1.9614&1.9602\\
		\midrule
		$\nu$ & 1598 & 9998\\
		\midrule
		$\epsilon_{\textit{STAT}}$  [\textsuperscript{o}C]& $\pm $1.9202 &  $ \pm$ 1.5259\\
		\bottomrule
		\bottomrule
	\end{tabular}
	\caption{Errore statistico delle due distribuzioni}
	\label{tab:errorestat}
\end{table}

Da Tab.\ref{tab:errorestat} si nota come un numero elevato di campioni garantisca un errore statistico molto contenuto, che risulta minore di \gls{symb:sigmaT} di un ordine di grandezza. Questo risultato deriva dalla presenza di $\sqrt{N}$ in Eq.(\ref{eq:sigmaTmean}), il cui valore ammonta a $\sim$ 40 per la prima serie, a $\sim$ 100 per la seconda. Per quanto riguarda il valore di \gls{symb:t95}, si osserva che tende al valore asintotico ($\nu \rightarrow \infty $) di 1.960 in entrambi i casi, quindi l'influenza di tale parametro sull'errore statistico è pari per le due serie.

\paragraph{Ulteriori analisi sui dati}
Si riportano i risultati di una analisi della Power Spectral Density del segnale di temperatura misurato. L'evoluzione temporale delle due serie di misure (Fig.\ref{fig:timeboth}) suggerisce che ci possa essere una dinamica che influenza l'andamento della temperatura in maniera macroscopica. 

\begin{figure}
	\centering
	\begin{subfigure}{0.5\textwidth}
		\centering
		\includegraphics[width=0.95\linewidth]{"../sperimentazione nei propulsori/time_short"}
		\caption{Serie corta}
		\label{fig:timeshort}
	\end{subfigure}%
\begin{subfigure}{0.5\textwidth}
	\centering
	\includegraphics[width=0.95\linewidth]{"../sperimentazione nei propulsori/time_long"}
	\caption{Serie lunga}
	\label{fig:timelong}
\end{subfigure}
\caption{Evoluzione temporale dei due segnali}
\label{fig:timeboth}
\end{figure}

Viene quindi calcolata la PSD mediante il metodo di Welch, con il segnale che viene diviso in 10 intervalli, ciascuno dei quali passato attraverso una finestra di Hamming, con overlap al 50 \%. Il risultato (Fig.\ref{fig:freqboth}) mostra che il segnale contiene componenti in frequenza di entità costante, come è atteso da un segnale costante nel tempo affetto da rumore assimilabile a rumore bianco. Tuttavia compare un picco di densità di potenza a bassa frequenza, che suggerisce la presenza di oscillazioni significative a bassa frequenza, legate possibilmente a una dinamica lenta. 


\begin{figure}
	\centering
	\begin{subfigure}{0.5\textwidth}
		\centering
	\includegraphics[width=0.95\linewidth]{"../sperimentazione nei propulsori/freq_short"}
	\caption{Serie corta}
	\label{fig:freqshort}
	\end{subfigure}%
	\begin{subfigure}{0.5\textwidth}
	\centering
\includegraphics[width=0.95\linewidth]{"../sperimentazione nei propulsori/freq_long"}
\caption{Serie lunga}
\label{fig:freqlong}
	\end{subfigure}
	\caption{PSD (one-sided) dei due segnali}
	\label{fig:freqboth}
\end{figure}
