\section{Stima dell'errore sistematico}

\paragraph{Dati e richieste}
Viene fornita la tabella di calibrazione statica di una termocoppia di tipo B, riportata in Tab. \ref{tab:calibrazione statica}. Si chiede di ricavare la relazione temperatura misurata-tensione mediante analisi di regressione, di determinare l'errore associato a tale processo, nonché il grado di adattamento del modello ai dati sperimentali. Infine si richiede di determinare l'errore totale associato alle misure di temperatura delle due serie di Sez.\ref{sez:misureT}.

\begin{table}[H]
	\centering
\begin{tabular}{c|c}
	\toprule\toprule
	\textbf{Tensione [mV]} & \textbf{Temperatura [°C]} \\ 
	\midrule\midrule
	0.786 & 400 \\ \midrule
	1.791 & 600 \\ \midrule
	2.431 & 700 \\ \midrule
	2.784 & 750 \\ \midrule
	3.158 & 800 \\ \midrule
	3.551 & 850 \\ \midrule
	3.963 & 900 \\ \midrule
	4.395 & 950 \\ \midrule
	4.844 & 1000 \\ \midrule
	5.311 & 1050 \\ \midrule
	5.793 & 1100 \\ \midrule
	6.290 & 1150 \\ \midrule
	6.800 & 1200 \\ \midrule
	7.326 & 1250 \\ \midrule
	7.866 & 1300 \\ \midrule
	8.418 & 1350 \\ \midrule
	8.979 & 1400 \\ \midrule
	9.549 & 1450 \\ \midrule
	10.124 & 1500 \\ \midrule
	10.704 & 1550 \\ \midrule
	11.286 & 1600 \\ 
	\bottomrule\bottomrule
\end{tabular}
\caption{Tabella di calibrazione statica}
\label{tab:calibrazione statica}
\end{table}

\begin{figure}
	\centering
	\includegraphics[width=0.7\linewidth]{"chapters/2-errsist/serie calibrazione"}
	\caption{Serie di dati della tabella di calibrazione statica}
	\label{fig:serie-calibrazione}
\end{figure}

\paragraph{Introduzione alla risoluzione}
Dalla rappresentazione di Fig.\ref{fig:serie-calibrazione} si nota che i dati sono caratterizzati da un andamento non lineare sull'intero campo di misura, mentre a partire da circa 6 mV si ha un andamento quasi lineare.
Di conseguenza, l'adozione di una legge di regressione non lineare, eventualmente di quarto ordine, potrebbe essere una scelta valida. Tuttavia, dato il ridotto numero di punti del problema, è opportuno delinare una strategia risolutiva che valuti più aspetti al fine di trovare la migliore soluzione. 


Si consideri la quantità \gls{symb:sigmaeps}, ossia l'errore di regressione, definita come:
\begin{equation}
	\sigma_{\varepsilon}^2= \frac{\Sigma_{i=1}^N \, (y_i - \tilde{y_i})^2}{N-G} 
\end{equation}
dove: 
\begin{itemize}
	\item $y_i$ è l'i-esimo valore misurato di \gls{symb:T};
	\item $\tilde{y_i}$ indica la stima di \gls{symb:T} ottenuta a ogni valore di tensione mediante la legge di regressione;
	\item $N$ indica il numero di coppie di dati usati per determinare i coefficienti di regressione. 
	\item $G$ indica il numero di coefficienti di regressione (vale $G = O + 1$, dove con $O$ si indica l'ordine del modello di regressione).
\end{itemize}
Utilizzare un numero elevato di punti permette di ridurre questo errore (aumento di $N$), così come un modello di ordine superiore riduce la somma degli scarti quadratici (numeratore della frazione), ma a parità di $N$ provoca la riduzione del denominatore con conseguente aumento dell'errore. 

Si evidenzia anche che il calcolo dell'errore sistematico (\gls{symb:errsist}) prevede di moltiplicare \gls{symb:sigmaeps} per la \gls{symb:t95} relativa al numero di gradi di libertà \gls{symb:nu}, ottenibile mediante $ \nu = N - G $ (il numero di gradi di libertà persi è pari al numero di coefficienti di regressione). A seguito di questa operazione si nota che la scelta di un numero limitato di punti per la regressione (N basso), associata a un modello di ordine alto, porta a \gls{symb:nu} molto bassi che comportano un significativo aumento di \gls{symb:errsist}.

Queste osservazioni motivano uno studio di ottimizzazione, con lo scopo di trovare l'ordine e il numero di coppie di dati di calibrazione che minimizzano globalmente l'errore sistematico. 
Poiché le due serie di dati presentano estremi differenti, lo studio viene proposto separatamente per entrambe.


\subsection{Risoluzione per serie corta}
Si presenta lo studio di ottimo per la prima serie di dati, svolto con le stesse modalità anche per la seconda. Il punto di partenza sono gli estremi della serie di dati, ossia la minima e massima \gls{symb:T} misurata, che assumono rispettivamente il valore di 953 \textsuperscript{o}C e di 1193 \textsuperscript{o}C. Da Tab.\ref{tab:calibrazione statica} si ricava che il minimo intervallo di dati su cui svolgere la regressione va da 950 \textsuperscript{o}C a 1200 \textsuperscript{o}C. A ogni iterazione questo secondo intervallo viene ampliato di un valore a sinistra dell'estremo inferiore e a destra dell'estremo superiore. Il più ampio intervallo possibile va da 400 \textsuperscript{o}C a 1550 \textsuperscript{o}C. 

Per ogni intervallo di dati di calibrazione sono determinati i coefficienti di regressione dal primo al quarto ordine (mediante funzione \textit{polyfit} di Matlab), poi vengono calcolate le stime $\tilde{y_i}$ per tutti gli ordini; successivamente si calcolano i $\sigma_{\varepsilon}$ e infine i valori di \gls{symb:errsist}. Per ogni interazione viene selezionato l'ordine del polinomio interpolante che garantisce il minimo \gls{symb:errsist}; dopo aver ripetuto l'analisi per tutti gli intervalli si individua il miglior errore sistematico globale. 


