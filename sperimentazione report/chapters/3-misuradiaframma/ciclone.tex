\section{Misura di portata mediante diaframma}
\subsection{Presentazione del banco prova, dati e richieste}
Viene assegnato un banco prova per misure di portata mediante diaframma normalizzato, rappresentato schematicamente in Fig.\ref{fig:schemabancoprova}. 

\begin{figure} [H]
	\centering
	\includegraphics[width=\linewidth]{chapters/3-misuradiaframma/schemabancoprova}
	\caption{Schema del banco prova}
	\label{fig:schemabancoprova}
\end{figure}

In particolare, il separatore a ciclone opera nelle seguenti condizioni nominali:
\begin{itemize}
	\item fluido di lavoro (fase gas): aria;
	\item portata di aria nelle condizioni operative nominali: 96 Nm\textsuperscript{3}/h;
	\item pressione di lavoro: 4 bar;
	\item temperatura di lavoro (ambiente): 300 K.
\end{itemize}

Al fine di misurare la portata d'aria si sceglie di utilizzare un diaframma normalizzato conforme alla Norma UNI EN ISO 5167-1, rappresentato in Fig.\ref{fig:schemadiaframma} e con le seguenti caratteristiche: 
\begin{itemize}
	\item \gls{symb:D} = 42 mm;
	\item \gls{symb:d} = 9.94 mm;
	\item Prese di pressione sulle flange.
\end{itemize}

\begin{figure} [H]
	\centering
	\includegraphics[width=0.7\linewidth]{chapters/3-misuradiaframma/schemadiaframma}
	\caption{Schema del diaframma (Fonte: UNI EN ISO 5167-1 Figura 5)}
	\label{fig:schemadiaframma}
\end{figure}

\paragraph{Richieste}
Si chiede di valutare la pressione minima di esercizio nel serbatoio di alimentazione e di indicare il trasduttore di pressione differenziale da usare sul banco prova per la misura di portata.


\subsection{Risoluzione}
Facendo riferimento a Fig.\ref{fig:schemabancoprova}, si denotino con \gls{symb:p}\textsubscript{1} e \gls{symb:p}\textsubscript{2} le pressioni (in Pa) a monte e a valle del diaframma. Si indichi con \gls{symb:T} la temperatura di esercizio (in K).

\paragraph{Ipotesi risolutive}
\begin{itemize}
	\item Le perdite di carico lungo il condotto e nelle valvole sono trascurate per mancanza di informazioni sull'impianto: di conseguenza \gls{symb:p}\textsubscript{1} è la pressione incognita del serbatoio di alimentazione, \gls{symb:p}\textsubscript{2} è la pressione operativa del separatore a ciclone.
	\item L'aria viene considerata come un gas perfetto, per cui \gls{symb:gamma} = \gls{symb:kappa} = 1.4, come indicato dalla Norma.
	\item Tutti i requisiti della Norma sono soddisfatti (Es. scabrosità del condotto, deformazione del diaframma, configurazione dello strumento).
	\end{itemize}

\paragraph{Riassunto dei dati e conversioni}
Si riporta un riassunto sintetico dei dati del problema, convertiti in unità del Sistema Internazionale dove necessario.
\begin{table}[H]
	\centering
	\begin{tabular}{c|c}
			\toprule
			\toprule
			\textbf{Dato} & \textbf{Valore} \\
			\midrule
			\midrule
			$D$ & 4.2e-2 m \\
			\midrule
			$d$ & 9.94e-3 m \\
			\midrule
			$p_2$ & 4.053e5 Pa\\
			\midrule
			$T$ & 300 K\\
			\midrule
			$\gamma$ & 1.4 \\
			\midrule
			$q_{\textit{vN}}$ & 96 Nm\textsuperscript{3}/h	\\
			\midrule
			$q_m$ & 0.0344 kg/s\\
			\bottomrule
			\bottomrule
	\end{tabular}
\end{table}
Al fine di convertire la portata volumetrica normalizzata in una portata massica si adotta la seguente espressione: 
\begin{equation}
	q_m = q_{\textit{vN}}\, \rho_n / 3600 
\end{equation}
con \gls{symb:rho}\textsubscript{N} = 1.293 kg/m\textsuperscript{3} la densità dell'aria a 273.15 K e 101325 Pa.

\paragraph{Svolgimento}