\section{Perdite per irraggiamento della misura di temperatura}
\subsection{Introduzione, dati e richieste}
La misura di temperatura del gas combusto è affetta da varie sorgenti di errore, tra cui spiccano le perdite per irraggiamento. 
Infatti il giunto caldo, essendo a temperatura significativamente più alta dell'ambiente circostante, è affetto da significativo scambio termico radiativo. Pertanto la temperatura misurata al giunto caldo non è quella effettiva del gas. 
Al fine di stimare tali perdite è possibile scrivere un bilancio energetico, che uguaglia il calore scambiato per irraggiamento a quello trasferito per convezione: 
\begin{equation}
	h \,(T_{\textit{GAS}}-T_G) = \sigma \varepsilon (T_G^4-T_A^4)
\end{equation}
dove compaiono le seguenti quantità:
\begin{itemize}
	\item \gls{symb:h} coefficiente di scambio termico convettivo [W/m\textsuperscript{2}];
	\item \gls{symb:Tgas}, \gls{symb:Tg}, \gls{symb:Ta} temperature del gas, del giunto e dell'ambiente [K];
	\item \gls{symb:sigma} costante di Stefan-Boltzmann [W/(m\textsuperscript{2}K\textsuperscript{4})];
	\item \gls{symb:emissivita} emissività della termocoppia.
\end{itemize}

Mediante tale equazione è immediato individuare le perdite radiative della termocoppia come:
\begin{equation}
	\Delta T_{\textit{RAD}} = T_{\textit{GAS}} -T_G = \sigma \varepsilon (T_G^4-T_A^4) / h \label{eq:deltaT}
\end{equation}

L'utilizzo di questa formula richiede la stima di \gls{symb:emissivita} e \gls{symb:h}: per la prima si fornisce un valore, mentre per la seconda viene fornita una correlazione semi-empirica (per cilindri immersi in flussi trasversali) nella forma:
\begin{equation}
	Nu = hD/k = CRe^nPr^m
\end{equation}
funzione di:
\begin{itemize}
	\item \gls{symb:k} conducibilità termica [W/(mK)];
	\item \gls{symb:Dt} diametro della termocoppia (modellata come un filo immerso in flusso ortogonale ad esso) [m];
	\item \gls{symb:re} numero di Reynolds definito come $Re = D_T V_G/\nu$, dove $\nu$ è la viscosità cinematica dell'aria [m\textsuperscript{2}/s];
	\item \gls{symb:Pr} numero di Prandtl;
	\item $C$, $n$, $m$ sono dei coefficienti che variano con $Re$ e $Pr$.
\end{itemize}

Vengono forniti i seguenti dati, riportati in Tab.\ref{tab:datiirr}.
\begin{table}[H]
	\centering
	\begin{tabular}{c|c}
		\toprule
		\toprule
		\textbf{Dato} & \textbf{Valore}\\
		\midrule
		\midrule
		\gls{symb:Ta} & 300 K\\
		\midrule
		\gls{symb:Tg} serie corta & 1356 K \\
		\midrule
		\gls{symb:Tg} serie lunga & 1432 K \\
		\midrule
		\gls{symb:Dt} & 3.5e-4 m \\
		\midrule
		\gls{symb:sigma} & 5.67e-8 W/(m\textsuperscript{2}K\textsuperscript{4})\\
		\midrule
		\gls{symb:emissivita} & 0.2 \\
		\midrule
		\midrule
		\textbf{Correlazione} & $ Nu = Nu(Pr,Re) $\\
		valida per & Re 1-35 \\
		\midrule
		$C$ & 0.8 \\
		$n$ & 0.384 \\
		$m$ & 0 \\
		\bottomrule
		\bottomrule
	\end{tabular}
\caption{Dati del problema}
\label{tab:datiirr}
\end{table}

\paragraph{Ipotesi}
Si adottano le seguenti ipotesi:
\begin{itemize}
	\item Utilizzare aria calda al posto dei gas combusti.
	\item \gls{symb:Vg} pari a 1 m/s.
	\item le perdite conduttive sono trascurabili.
	\item \gls{symb:Tg} è pari alla temperatura media misurata nelle due serie di campioni.
\end{itemize}

\paragraph{Richieste}
Si chiede di stimare le perdite radiative alla velocità fornita e quindi la temperatura del gas. Si ripeta la stima per \gls{symb:Vg} pari a 50 m/s. 

\subsection{Risoluzione}
La risoluzione del problema prevede di utilizzare Eq.(\ref{eq:deltaT}) per calcolare \gls{symb:Tgas}, da cui poi si ricava \gls{symb:deltaTrad}. Poiché \gls{symb:h} dipende da \gls{symb:Tgas} attraverso \gls{symb:re}, \gls{symb:Pr} e altre grandezze, si risolve il problema iterativamente a partire da un valore ragionevole di \gls{symb:Tgas}. Con tale valore si determinano le proprietà dell'aria e si trova un valore di \gls{symb:h}, che a sua volta permette di determinare il valore aggiornato di \gls{symb:Tgas}. Si arresta il processo quando due iterazioni successive differiscono di meno di 1 K.
Tale processo viene ripetuto analogamente per serie corta e serie lunga di dati. 
Le proprietà dell'aria vengono ottenute per interpolazione lineare dei dati nella tabella fornita. 


\paragraph{Risultati a 1 m/s}
In Tab.\ref{tab:calcolishort1} si riportano integralmente i calcoli impiegati per raggiungere il risultato relativo alla serie corta con una velocità di 1 m/s. 

\begin{table}[H]
	\centering
	\begin{tabular}{c|c|c|c|c|c|c|c|c}
		\toprule
		\toprule
		\textbf{Iterazione} & $\bm{T_{\textit{GAS}}}$ [K]& $\bm{k}$ [W/(mK)] & $\bm{\nu}$ [m\textsuperscript{2}/s] & $\bm{Re}$ & $\bm{C}$ & $\bm{n}$ & $\bm{h}$ [W/m\textsuperscript{2}] & $\bm{\Delta T_{\textit{ITER}} \%}$\\
		\midrule
		\midrule
		0 & 1500 & 0.0946 & 2.29e-4 & 1.53 & 0.8 & 0.384 & 254.44 & - \\
		\midrule
		1 & 1506 & 0.0949 & 2.31e-4 & 1.52 & 0.8 & 0.384 & 254.67 & 0.41 \% \\
		\midrule
		2 & 1506 & & &  &  &  &  &  -0.009 \% \\
		\bottomrule
		\bottomrule	
	\end{tabular}
	\caption{Calcoli per la serie corta}
	\label{tab:calcolishort1}
\end{table}

In Tab.\ref{tab:risultatishort1} vengono riportati sinteticamente i risultati relativi alla serie corta di dati, a 1 m/s.

\begin{table}[H]
	\centering
	\begin{tabular}{c|c}
		\toprule
		\toprule
		& $\bm{T}$ [K]\\
		\midrule
		\midrule
		\gls{symb:Tg} & 1356 \\
		\midrule
		\gls{symb:Tgas} & 1506 \\
		\midrule
		$\Delta T_{\textit{RAD}}$ & 150 \\
		\bottomrule
		\bottomrule	
	\end{tabular}
	\caption{Risultati per la serie corta}
	\label{tab:risultatishort1}
\end{table}



In Tab.\ref{tab:risultatilong1} vengono riportati sinteticamente i risultati relativi alla serie lunga di dati, sempre a 1 m/s.

\begin{table}[H]
	\centering
	\begin{tabular}{c|c}
		\toprule
		\toprule
		         & $\bm{T}$ [K]\\
		\midrule
		\midrule
		\gls{symb:Tg} & 1432 \\
		\midrule
		\gls{symb:Tgas} & 1616 \\
		\midrule
		$\Delta T_{\textit{RAD}}$ & 184 \\
		\bottomrule
		\bottomrule	
	\end{tabular}
\caption{Risultati per la serie lunga}
\label{tab:risultatilong1}
\end{table}

Si osserva che in entrambi i casi non si esce dall'intervallo di \gls{symb:re} per cui è valida la correlazione fornita. Inoltre si nota che le perdite radiative sono più significative quando la temperatura del giunto caldo è più alta. Si pone l'attenzione sull'elevato errore dovuto alle ingenti perdite per radiazione, che è di due ordini di grandezza superiore all'errore totale legato alla misura. 

\paragraph{Risultati a 50 m/s}
Lo stesso procedimento è seguito per individuare \gls{symb:Tgas} nel flusso a 50 m/s. Si utilizza un valore iniziale di 1400 K, scelto inferiore al risultato a 1 m/s di 1506 K, in quanto ci si attende che la convezione forzata sia agevolata dal flusso ad alta velocità. Utilizzando tale temperatura è calcolato un valore di \gls{symb:re} di 87 per la serie corta. La correlazione utilizzata inizialmente non è più valida. Un risultato analogo si ottiene con la serie lunga, in cui la temperatura iniziale è imposta a 1470 K e il \gls{symb:re} è pari a 80.

Si sceglie di sostituirla con la seguente correlazione (Hilpert), valida per cilindri immersi in flussi (in direzione ortogonale al corpo) con \gls{symb:re} da 40 a 4000:
\begin{equation}
	Nu = 0.68 Re^{0.466}Pr^{1/3}
\end{equation}



A ogni iterazione è necessario verificare che si rimanga nel campo di validità della correlazione. Quest'ultima richiede di individuare la temperatura di film (\gls{symb:Tf}), ottenuta mediante la relazione: 
\begin{equation}
	T_F=\frac{T_G+T_{\textit{GAS}}}{2}
\end{equation}

Si riportano soltanto i risultati ottenuti per la serie corta, in Tab.\ref{tab:risultatishort50}, e per la serie lunga, in Tab.\ref{tab:risultatilong50}. La validità della correlazione è verificata in tutti i casi.  

\begin{table}[H]
	\centering
	\begin{tabular}{c|c}
		\toprule
		\toprule
		& $\bm{T}$ [K]\\
		\midrule
		\midrule
		\gls{symb:Tg} & 1356 \\
		\midrule
		\gls{symb:Tgas} & 1387 \\
		\midrule
		$\Delta T_{\textit{RAD}}$ & 31 \\
		\bottomrule
		\bottomrule	
	\end{tabular}
	\caption{Risultati per la serie corta}
	\label{tab:risultatishort50}
\end{table}

\begin{table}[H]
	\centering
	\begin{tabular}{c|c}
		\toprule
		\toprule
		& $\bm{T}$ [K]\\
		\midrule
		\midrule
		\gls{symb:Tg} & 1432 \\
		\midrule
		\gls{symb:Tgas} & 1471 \\
		\midrule
		$\Delta T_{\textit{RAD}}$ & 39 \\
		\bottomrule
		\bottomrule	
	\end{tabular}
	\caption{Risultati per la serie lunga}
	\label{tab:risultatilong50}
\end{table}

Come predetto, l'aumento della velocità del flusso porta a un significativo aumento dello scambio termico convettivo, con conseguente riduzione delle perdite per irraggiamento. Questo fenomeno è confermato dall'aumento di \gls{symb:h}, i cui valori sono riportati per i 4 casi in Tab.\ref{tab:risultatih}.

\begin{table}[H]
	\centering
	\begin{tabular}{c|c}
		\toprule
		\toprule
		\textbf{Caso}& $\bm{h}$  [W/m\textsuperscript{2}] \\
		\midrule
		\midrule
		Serie corta, 1 m/s & 255 \\
		\midrule
		Serie corta, 50 m/s & 1226 \\
		\midrule
		Serie lunga, 1 m/s & 259 \\
		\midrule
		Serie lunga, 50 m/s & 1233 \\
		\bottomrule
		\bottomrule	
	\end{tabular}
	\caption{Valori di \gls{symb:h} nei 4 casi}
	\label{tab:risultatih}
\end{table}

\paragraph{Sensitività alla variazione di $\bm{\varepsilon}$}
Si presenta il risultato di uno studio di sensitività alla variazione di \gls{symb:emissivita}. Questa è una delle grandezze la cui misura è complessa e affetta da significativo errore. Lo studio è proposto per la serie corta a 1 m/s in 5 valori di \gls{symb:emissivita}.

\begin{figure} [H]
	\centering
	\includegraphics[width=0.7\linewidth]{"../sperimentazione nei propulsori/varepsilon"}
	\caption{Variazione di \gls{symb:emissivita}}
	\label{fig:variazioneepsilon}
\end{figure}

Si osserva come una variazione modesta di \gls{symb:emissivita} porti a grandi variazioni di $\Delta T_{\textit{RAD}}$. Si sottolinea quindi la necessità di caratterizzare il suo valore con elevata precisione. L'andamento osservato è lineare, quanto meno nel campo di valori studiati.

\paragraph{Sensitività a variazione di $\bm{V_G}$}
Si riportano i risultati ottenuti a varie velocità del gas. A ognuna è associata la correlazione opportuna nella forma:
\begin{equation}
	Nu = C Re^{n}Pr^{1/3} \label{eq:corrnew}
\end{equation}
Si riportano i valori di $C$ e $n$ per vari intervalli di \gls{symb:re}. Per il caso a 1 m/s si utilizza una correlazione diversa da quella iniziale. Come in precedenza, si utilizza \gls{symb:Tf} per il calcolo delle proprietà dell'aria.

\begin{table}[H]
	\centering
	\begin{tabular}{c|c|c}
		\toprule
		\toprule
		\textbf{Intervallo}&  $\bm{C}$ &  $\bm{n}$\\
		\midrule
		\midrule
		0.4 - 4 & 0.989 & 0.33\\
		\midrule
		4 - 40 & 0.911 &0.385\\
		\midrule
		40 - 4000 & 0.683 &0.466\\
		\midrule
		4000 - 40000 & 0.193 &0.618\\
		\bottomrule
		\bottomrule	
	\end{tabular}
	\caption{Indici della correlazione per vari $Re$}
	\label{tab:RErange}
\end{table}

\begin{figure}[H]
	\centering
	\includegraphics[width=0.7\linewidth]{"../sperimentazione nei propulsori/varV"}
	\caption{Variazione di \gls{symb:Vg}}
	\label{fig:varv}
\end{figure}

Si osserva che all'aumentare di \gls{symb:Vg} inizialmente si ottiene una significativa riduzione delle perdite radiative. All'aumentare della velocità si tende a raggiungere un valore di $\Delta T_{\textit{RAD}}$ ridotto, pari ai 31 K precedentemente individuati. Tuttavia la velocità che deve avere il flusso per ridurre le perdite diventa rapidamente elevata, dando origine ad altro errore, per esempio quello legato agli effetti di comprimibilità. Si nota, infine, che il valore a 1 m/s con la prima correlazione è molto simile a quello a pari velocità con la correlazione di Eq.(\ref{eq:corrnew}).

\subsection{Conclusioni}
L'impatto delle perdite radiative è stimato e si osserva che queste provocano un significativo errore nella misura di temperatura dei gas combusti. La perdita è tanto crescente quanto è più alta la temperatura del giunto caldo, la quale causa un maggiore scambio termico per irraggiamento. Le perdite radiative sono di vari ordini di grandezza superiori all'errore totale precedentemente calcolato. Si può ridurre la perdita andando ad aumentare la velcoità del flusso, se fattibile, oppure diminuendo l'emissività della termocoppia. Il metodo solitamente utilizzato prevede la schermatura dello strumento. 