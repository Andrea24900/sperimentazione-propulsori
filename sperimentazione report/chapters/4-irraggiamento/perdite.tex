\section{Perdite per irraggiamento della misura di temperatura}
\subsection{Introduzione, dati e richieste}
La misura di temperatura del gas combusto è affetta da varie sorgenti di errore, tra cui spiccano le perdite per irraggiamento. 
Infatti il giunto caldo, essendo a temperatura significativamente più alta dell'ambiente circostante, irradia molto calore. Pertanto la temperatura misurata al giunto caldo non è quella effettiva del gas. 
Al fine di stimare tali perdite è possibile scrivere un bilancio energetico, che uguaglia il calore scambiato per irraggiamento a quello trasferito per convezione: 
\begin{equation}
	h \,(T_{\textit{GAS}}-T_G) = \sigma \varepsilon (T_G^4-T_A^4)
\end{equation}
dove compaiono le seguenti quantità:
\begin{itemize}
	\item \gls{symb:h} coefficiente di scambio termico convettivo [W/m\textsuperscript{2}];
	\item \gls{symb:Tgas}, \gls{symb:Tg}, \gls{symb:Ta} temperature del gas, del giunto e dell'ambiente [K];
	\item \gls{symb:sigma} costante di Stefan-Boltzmann [W/(m\textsuperscript{2}K\textsuperscript{4})];
	\item \gls{symb:emissivita} emissività della termocoppia.
\end{itemize}

Mediante tale equazione è immediato individuare le perdite radiative della termocoppia come:
\begin{equation}
	\Delta T_{\textit{RAD}} = T_{\textit{GAS}} -T_G = \sigma \varepsilon (T_G^4-T_A^4) / h \label{eq:deltaT}
\end{equation}

L'utilizzo di questa formula richiede la stima di \gls{symb:emissivita} e \gls{symb:h}: per la prima si fornisce un valore, mentre per la seconda viene fornita una correlazione semi-empirica nella forma:
\begin{equation}
	Nu = hD/k = CRe^nPr^m
\end{equation}
funzione di:
\begin{itemize}
	\item \gls{symb:k} conducibilità termica [W/(mK)];
	\item \gls{symb:Dt} diametro della termocoppia (modellata come un filo immerso in flusso ortogonale ad esso) [m];
	\item \gls{symb:re} numero di Reynolds definito come $Re = D_T V_G/\nu$, dove $\nu$ è la viscosità cinematica dell'aria [m\textsuperscript{2}/s];
	\item \gls{symb:Pr} numero di Prandtl;
	\item $C$, $n$, $m$ sono dei coefficienti che variano con $Re$ e $Pr$.
\end{itemize}

Vengono forniti i seguenti dati, riportati in Tab.\ref{tab:datiirr}.
\begin{table}[H]
	\centering
	\begin{tabular}{c|c}
		\toprule
		\toprule
		\textbf{Dato} & \textbf{Valore}\\
		\midrule
		\midrule
		\gls{symb:Ta} & 300 K\\
		\midrule
		\gls{symb:Tg} serie corta & 1356 K \\
		\midrule
		\gls{symb:Tg} serie lunga & 1432 K \\
		\midrule
		\gls{symb:Dt} & 3.5e-4 m \\
		\midrule
		\gls{symb:sigma} & 5.67e-8 W/(m\textsuperscript{2}K\textsuperscript{4})\\
		\midrule
		\gls{symb:emissivita} & 0.2 \\
		\midrule
		\midrule
		\textbf{Correlazione} & $ Nu = Nu(Pr,Re) $\\
		valida per & Re 1-35 \\
		\midrule
		$C$ & 0.8 \\
		$n$ & 0.384 \\
		$m$ & 0 \\
		\bottomrule
		\bottomrule
	\end{tabular}
\caption{Dati del problema}
\label{tab:datiirr}
\end{table}

\paragraph{Ipotesi}
Si adottano le seguenti ipotesi:
\begin{itemize}
	\item Utilizzare aria calda al posto dei gas combusti.
	\item \gls{symb:Vg} pari a 1 m/s.
	\item le perdite conduttive sono trascurabili.
	\item \gls{symb:Tg} è pari alla temperatura media misurata nelle due serie di campioni.
\end{itemize}

\paragraph{Richieste}
Si chiede di stimare le perdite radiative alla velocità fornita e quindi la temperatura del gas. Si ripeta la stima per \gls{symb:Vg} pari a 50 m/s. 

\subsection{Risoluzione}
La risoluzione del problema prevede di utilizzare Eq.(\ref{eq:deltaT}) per calcolare \gls{symb:Tgas}, da cui poi si ricava $\Delta T_{\textit{RAD}}$. Poiché \gls{symb:h} dipende da \gls{symb:Tgas} attraverso \gls{symb:re}, \gls{symb:Pr} e altre grandezze, si risolve il problema iterativamente a partire da un valore ragionevole di \gls{symb:Tgas}. Con tale valore si determinano le proprietà dell'aria e si trova un valore di \gls{symb:h}, che a sua volta permette di determinare il valore aggiornato di \gls{symb:Tgas}. Si arresta il processo quando due iterazioni successive differiscono di meno di 1 K.
Tale processo viene ripetuto analogamente per serie corta e serie lunga di dati. 
Le proprietà dell'aria vengono ottenute per interpolazione lineare dei dati in App.\ref{app:aria}.

In Tab. si riportano integralmente i calcoli impiegati per raggiungere il risultato relativo alla serie corta con una velocità di 1 m/s. 



In Tab. vengono riportati sinteticamente i risultati relativi alla serie lunga di dati, sempre a 1 m/s.


Si osserva che in entrambi i casi non si esce dall'intervallo di \gls{symb:re} per cui è valida la correlazione fornita. Inoltre si nota che le perdite radiative sono più significative quando la temperatura del giunto caldo è più alta. 

