%---- packages: ---------------

% \usepackage[LGR,T1]{fontenc} % per la codifica dei font
%                              % T1 per i documenti in lingua inglese
%                              % LGR consente la scrittura in greco
% \usepackage[utf8]{inputenc} % usato per gestire l'input di caratteri utf8
% \usepackage[english]{babel} % offre supporto per la localizzazione dei testi, formattazione e 
%                             % convenzioni tipografiche specifiche per diverse lingue
% \usepackage{substitutefont}
% \usepackage{newtxtext}
% \usepackage{newtxmath}
% \substitutefont{LGR}{\rmdefault}{Tempora}
\usepackage{siunitx}
\usepackage[hidelinks]{hyperref} % per collegamenti a siti

\usepackage[            
    nogroupskip,                
    acronym,                
    nomain,                     
    nopostdot,                  
    sort=standard,
    nonumberlist,
    automake,            
    xindy,
    stylemods={longbooktabs},
]
{glossaries-extra}


\usepackage{amsmath}
\usepackage{lipsum}
\usepackage{parskip}
\usepackage{graphicx} % Required for inserting images
\usepackage{geometry}
\usepackage{bm}
\usepackage{amsmath,amssymb} % per l'aggiunta di formule e simboli matematici
\usepackage{float}
\usepackage{caption}
\usepackage{subcaption}
\usepackage{braket}
\usepackage{array}
\usepackage{booktabs}
\usepackage{enumitem}
\usepackage{titlesec}
\usepackage[style=numeric, backend=biber,defernumbers=true]{biblatex}
\usepackage[bottom]{footmisc} % per le note a piè di pagina
\usepackage{multirow} % per unire celle nelle tabelle
\usepackage{eurosym} % per simbolo di euro fatto meglio
\usepackage{gensymb} % per simbolo dei gradi
\usepackage{textcomp}
\usepackage{fancyhdr}
\usepackage{titlesec}
\usepackage{cleveref}
\usepackage{tocloft}
\usepackage{longtable}


\setlength{\cfttabnumwidth}{3em}    % imposta spazio indice tabelle (es. 10.11)
\setlength{\cftfignumwidth}{3em}

\numberwithin{equation}{section}  % Imposta la numerazione delle equazioni con il numero della section
\renewcommand\theequation{\thesection.\arabic{equation}}  % Imposta il formato (A.B) per la numerazione delle equazioni
\numberwithin{table}{section}  % Numerazione delle tabelle con il numero della section
\renewcommand\thetable{\thesection.\arabic{table}}  % Formato (A.B) per la numerazione delle tabelle

\numberwithin{figure}{section}  % Numerazione delle figure con il numero della section
\renewcommand\thefigure{\thesection.\arabic{figure}}  % Formato (A.B) per la numerazione delle figure

\usepackage{appendix} % per inserire appendici

\usepackage[table]{xcolor} % per colorare celle di tabelle

\pagestyle{fancy}
\fancyhf{} % Cancella tutti gli stili predefiniti
\renewcommand{\headrulewidth}{0.4pt}
% \renewcommand{\footrulewidth}{0.4pt}
\fancyhead[L]{\slshape \rightmark}     % L: left / R: right / C: central
\fancyfoot[C]{\thepage}
\setlength{\headheight}{13.6pt}
\addtolength{\topmargin}{-1.6pt}




% PATH FOR IMPORTING IMAGES
% \graphicspath{{chapters/}}

% PACKAGE FOR ACRONYM LIST
%\usepackage[acronyms, nonumberlist, nopostdot, toc]{glossaries}

% PACKAGE FOR LISTS, OPTIONS
\usepackage{array}
\usepackage[nottoc,notbib]{tocbibind}

% non toccare i margini
\geometry{
    top    = 2.5cm,  % Margine superiore
    bottom = 2.5cm,  % Margine inferiore
    left   = 2.5cm,  % Margine sinistro
    right  = 2.5cm   % Margine destro
}

\newcolumntype{C}[1]{>{\centering\arraybackslash}m{#1}}


